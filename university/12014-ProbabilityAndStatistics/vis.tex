\documentclass{report}

\usepackage{amsthm}
\usepackage{amsmath}
\usepackage{amssymb}
\usepackage{pgfplots}
\usepackage{enumitem}
\usepackage{stackengine}
\usepackage{xcolor}

\newcommand{\Dbar}{\stackinset{l}{0.1ex}{c}{}{\rule{0.33em}{0.3pt}}{D}}
\newcommand{\rjesenje}{\begin{flushleft}\it Rjesenje:\end{flushleft}}
\newcommand{\onespace}{\vspace{1pc}}
\newcommand{\highlight}[1]{%
  \colorbox{green!42}{$\displaystyle#1$}}

\theoremstyle{plain}
\newtheorem{thm}{Zadatak}[chapter] % reset theorem numbering for each chapter

\setcounter{MaxMatrixCols}{20}

\begin{document}

\centerline{\sc \large Vjerojatnost i statistika}
\vspace{.5pc}
\centerline{\sc Kratki pregled gradiva obra\scriptsize\Dbar\normalsize enih na Visokom učilištu Algebra}
\centerline{\it (Neslužbeni dokument, Tomislav Kucar, 15.9.2020.)}
\vspace{2pc}


\section{Vjerojatnost i uvjetna vjerojatnost}
\begin{thm}Strijelac gada metu najvise 4 puta, a igra se prekida kad pogodi.
    \begin{enumerate}[label=(\alph*)]
        \item Odredite prostor elementarnih dogadaja
        \item Ako je vjerojatnost pogodtka neovisna o prethodima i iznosi 0.4 
        odredite vjerojatnost da je strijelac pogodio metu u trecem pokusaju
        \item Odredite vjerojatnost da strijelac nije pogodio metu
    \end{enumerate}
\end{thm} 
\rjesenje

(a) H - hit, M - miss : $\Omega = \{H, MH, MMH, MMMH, MMMM\}$

\onespace

(b) Posto je vjerojatnost pogodtka neovisna o prethodnima: \vspace{0.5pc}

$P(MMH) = P(M) * P(M) * P(H)$ \vspace{0.5pc}

I uvrstavamo $P(H) = 0.4$ : \vspace{0.5pc}

$P(MMH) = 0.6 * 0.6 * 0.4 = 0.144$

\onespace

(c) $P(MMMM) = 0.6^4 = 0.1296$


\onespace
\begin{thm}U 6 kutija na slucajni nacin se rasporeduju 4 razlicite kuglice. 
    Kolika je vjerojatnost da ce u prve 4 kutije biti tocno po jedna kuglica?\end{thm} 
\rjesenje

Svaka od 4 kuglice moze pasti u bilo koju od 6 kutija, sto znaci da ukupno ima
$6^4$ mogucnosti. Nadalje prva kuglica mora upasti u jednu od 4 kutije, druga onda
mora upasti u jednu od 3 kutije, treca u jednu od 2 kutije, i zadnja kuglica u 
zadnju kutiju, dakle $4*3*2*1$. Znaci rjesenje je $\frac{4!}{6^4}$

\newpage

\begin{thm}Iva i Martin igraju igru: svaki ce zamisliti broj izmedu 1 i 20 (ukljuceni). 
    Kolika je vjerojatnost da je razlika tih brojeva veca od 10?
\end{thm} 
\rjesenje

X = Iva, Y = Martin

Njihovi brojevi se mroaju razlikovati za vise od 10. Posto ne pise tko je uzeo kojih
bitno je samo da njihova razlika mora biti strogo veca od 10: \vspace{0.5pc}

\begin{align*}
    \underline{|X-Y| > 10} \\
    x - y \geq 10 \\
    -x + y \geq 10 \\
    \\
    1^{\circ}\; y \leq 10 - x\\
    2^{\circ}\; y \geq 10 + x
\end{align*}

Nacrtajmo graf: \vspace{0.5pc}

\begin{tikzpicture}
    \begin{axis}[ 
      xlabel={$X = Iva$},
      ylabel={$Y = Martin$},
      xmin=0,
      xmax=20,
      ymin=0,
      ymax=20,
    ] 
      \addplot[mark=none, color=blue, fill=blue, fill opacity=0.05] coordinates { (0,10) (10,20) (0,20) }; 
      \addlegendentry{Prvi slucaj}
      \addplot[mark=none, color=red, fill=red, fill opacity=0.05] coordinates { (10,0) (20,10) (20,0) }; 
      \addlegendentry{Drugi slucaj}
    \end{axis}
  \end{tikzpicture}

Da bi dobili geometrijsku vjerojatnost trebamo izracunati povrsinu osjencanih 
trokuta kroz povrsinu kvadrata.

\begin{align*}
    \frac{P(\triangle_1)+P(\triangle_2)}{P(\square)} = \frac{\frac{10*10}{2}*2}{20*20} = \frac{100}{400} = \frac{1}{4}
\end{align*}


\newpage

\begin{thm}Maja i Marko igraju igru: Maja baca novcic i dobiva ako padne glava. 
    Marko baca kocku i dobiva ako padne neparan broj.
    \begin{enumerate}[label=(\alph*)]
        \item Je li igra fer?
        \item Jesu li dogadaji A = \{ Marko je dobio \} i B = \{ Maja je dobila \} neovisni?
    \end{enumerate}
\end{thm} 
\rjesenje

(a) Igra je fer jer oboje imaju istu vjerojatnost dobiti.

\onespace

(b) A = Maja je dobila, B = Marko je dobio

\onespace

Znamo da su dogadaji nezavisni ako vrijedi $P(A\cap B) = P(A) * P(B)$

\onespace

S obzirom da je mali vjerojatnosni prostor mozemo ispisati sve moguce dogadaje, P - pismo, G - glava:

\begin{align*}
    (P,1), (P,2), (P,3), (P,4), (P,5), (P,6) \\
    (G,1), (G,2), (G,3), (G,4), (G,5), (G,6)
\end{align*}

Maja dobiva ako padne glava, a marko dobiva ako je pao neparan broj,
dakle (G,1), (G,3), (G,5), i takvih dogadaja imamo ukupno 3 od 12 mogucih:

\onespace
 
$P(A\cap B) = \frac{3}{12}$

\onespace

I onda mozemo uvrsiti:

\begin{align*}
        P(A\cap B) &= P(A) * P(B) \\
      \frac{3}{12} &= \frac{1}{2} * \frac{1}{2} \\
       \frac{1}{4} &= \frac{1}{4} 
\end{align*}

Dogadaji su nezavisni.


\onespace
\begin{thm}Dva stroja proizvode cavle. Na prvom stroju se proizvede 40\% cavala, 
    pri cemu ostaje 3\% neispravnih, a na drugom stroju preostane 1\% neispravnih.
    \begin{enumerate}[label=(\alph*)]
        \item Kolika je vjerojatnost da slucajno odabran cavao bude neispravan?
        \item Ako je cavao los, kolika je vjerojatnost da je izraden na stroju 1?
    \end{enumerate}
\end{thm}
\rjesenje 
(a) Iz zadatka vidimo slijedece hipoteze:
\begin{align*}
    H_1 = \{ \text{Cavao je iz 1. stroja}\}, \; P(H_1) = 0.4 \\
    H_2 = \{ \text{Cavao je iz 2. stroja}\}, \; P(H_2) = 0.6
\end{align*}

\newpage

Moramo odrediti i vjerojatnost da je cavao neispravan a bio je na prvom stroju; 
$P(A|H_1)=3\%$ i ako je dosao iz drugog stroja $P(A|H_1)=1\%$.

\onespace

Zanima nas vjerojatnost od A = \{ cavao je neispravan \} sto cemo izracunati formulom
potpune vjerojatnosti:

\begin{align*}
    P(A) = P(A|H_1) * P(H_1) + P(A|H_2) * P(H_2) = 0.4 * 0.03 + 0.6 * 0.01 = 0.018
\end{align*}

\onespace

(b) Ako je cavao los, vjerojatnost da je izraden na prvom stroju?

\begin{align*}
    P(H_1|A) = \frac{P(A|H_1) * P(H_1)}{P(A)} = \frac{0.03 * 0.4}{0.018} = \frac{0.012}{0.018} = \frac{2}{3} = 66\%
\end{align*}

Ako je cavao los 66\% da je dosao iz stroja 1.

\section{Diskretne i neprekinute slucajne varijable}
\onespace
\begin{thm}Bacamo dvije kocke i zbrajamo brojeve na njima. Ako je zbroj djeljiv s 3, 
    dobivamo X kuna, inace gubimo 5 kuna. Odredite X tako da igra bude fer. 
    (ocekivani rezultat 0)\end{thm} 
\rjesenje

$X \sim \begin{pmatrix}2& 3& 4& 5& 6& 7& 8& 9& 10& 11& 12 \\ \frac{1}{36}& \frac{2}{36}& \frac{3}{36}& \frac{4}{36}& \frac{5}{36}& \frac{6}{36}& \frac{5}{36}& \frac{4}{36}& \frac{3}{36}& \frac{2}{36}& \frac{1}{36} \end{pmatrix} $

\onespace

X je nasa slucajna varijabla. Gornji red predstavlja vrijednosti, a doljni vjerojatnosti.
Vjerojatnosti smo dobili odredivanjem na koliko nacina mozemo dobiti odredenu vrijednost
i djeljeci sa ukupnim brojem mogucnosti $6^2$ tj. 36. Nadalje racunamo:

\onespace

$P(\text{zbroj je djeljiv s 3})=\frac{2}{36}+\frac{5}{36}+\frac{4}{36}+\frac{1}{36}=\frac{1}{3},
P(\text{zbroj nije djeljiv s 3})=\frac{2}{3}$

\onespace

Ako se ostvarilo da je broj djeljiv s 3 ostavarujemo vrijednost od X kuna, a ako
se nije ostvarilo onda gubimo 5 kuna.

\onespace

$Y \sim \begin{pmatrix} X& -5 \\ \frac{1}{3}& \frac{2}{3} \end{pmatrix} $

\onespace

Ocekivanje od Y mora biti 0 da bi igra bila postena.

\begin{align*}
    E(Y) &= 0\\
    X * \frac{1}{3} -5 * \frac{2}{3} &= 0 \\
    X &= 10
\end{align*}

Dakle ako si isplacujemo 10 kuna kad dobijemo, a gubimo 5 kuna kad izgubimo, 
dugorocno cemo biti cemo na nuli.

\onespace
\begin{thm}Ako nasumicno odgovaramo na 5 pitanja od kojih svako ima 3 ponudena odgovora,
    kolika je vjerojatnost da cemo na 3 odgovoriti tocno?\end{thm} 
\rjesenje
    
Ovdje imamo binomnu razdiobu sa 5 "pokusa" i vjerojatnoscu uspjeha 1/3 
(pogadamo jedan od tri moguca odgovora). $X \sim B(5, \frac{1}{3}) $

\onespace

Zanima nas vjerojatnost da smo pogodili tocno 3 pitanja (od 5). 
Prema formuli za binomnu razdiobu to je:

\begin{align*}
    P(X=3)=\binom{5}{3}*\left(\frac{1}{3}\right)^3*\left(1-\frac{1}{3}\right)^2 = 0.164
\end{align*}

Vjerojatnost da cemo pogoditi 3 od 5 pitanja je 16.4\%.

\begin{flushleft}\it Dodatno:\end{flushleft}

\begin{align*}
    E(X)=n*p=5*\frac{1}{3}=1.66
\end{align*}

\onespace
\begin{thm}Odredite funkciju distribucije slucajne varijable X s gustocom
    \begin{align*}
        f(x) = \left\{\begin{matrix} x, &0 < x\leq 1 \\ 
        \frac{1}{2x}, & 1 < x \leq e \\ 
        0, & inace \end{matrix}\right.
    \end{align*}
\end{thm} 
\rjesenje

Kad pricamo o neprekidnim slucajnim varijablama uvijek gledamo intervale,
jer je sansa za odabir odredene tocke je jednaka nuli.

\onespace

\begin{align*}
    F(a)&=P(X\leq a)=\int\limits_{-\infty}^a f(x)dx \\
    F(x)&=\left\{
        \begin{matrix} 
            a, &0 < x \\ 
            *, & 0 \leq a \leq 1 \\ 
            **, & 1 < a \leq e
        \end{matrix}1\right.
\end{align*}

Racunamo:
\begin{align*}
    * &= \int\limits_{-\infty}^a f(x)dx = \int\limits_0^a x dx = \frac{x^2}{2}\Biggr|_{0}^{a}=\frac{a^2}{2} \\
    ** &= \int\limits_{-\infty}^a f(x)dx = \int\limits_0^a f(x) dx = \int\limits_0^1 f(x) dx + \int\limits_1^e f(x) dx = \int\limits_0^1 x dx + \int\limits_1^e \frac{1}{2x} dx \\
    &= \frac{x^2}{2}\Biggr|_{0}^{1} +\frac{1}{2} * ln(x) = \frac{1}{2} + \frac{1}{2} * ln(a) - ln(1) = \frac{1}{2} + \frac{1}{2} * ln(a) 
\end{align*}

I uvrstimo:
\begin{align*}
    F(x)=\left\{
        \begin{matrix} 
            a, &0 < x \\ 
            \frac{a^2}{2}, & 0 \leq a \leq 1 \\ 
            \frac{1}{2}+\frac{1}{2}+ln(a), & 1 < a \leq e
        \end{matrix}1\right.
\end{align*}

\begin{flushleft}\it Dodatno:\end{flushleft}
\begin{align*}
    f \; gustoca \;&-> distribucija \;F :\text{integrirati} \\
    F \; distribucija \; &-> gustoca \; f :\text{derivirati}
\end{align*}

\onespace
\begin{thm}Na nekom ispitu bodovi studenata su normalno distribuirani s ocekivanjem
    od 68 bodova i disperzijom (varijancom) od 196 bodova. Za prolaz je bilo potrebno
    51 bod, a za ocjenu odlican 89 bodova.
    \begin{enumerate}[label=(\alph*)]
        \item Odredite koliki postotak studenata nije polozio ispit?
        \item Koliki postotak studenata je dobio ocjenu odlican?
    \end{enumerate}
\end{thm} 
\rjesenje

(a) 

\begin{align*}
& X \sim N\left(68, 14^2\right) \\
& P(X<51) = \left(\frac{x-68}{14}\leq \frac{51-68}{14}\right) \\
& Y \sim \frac{x-61}{14} \sim N(0,1) \\
& P(Y\leq-1.2142) = \Phi(-1.2142) = 1-\Phi(-1.2142) \\
&= 1-\frac{1}{2}-\frac{1}{2}\Phi^*(1.2142) = 0.5-\frac{1}{2}*0.785=0.1075
\end{align*}

(b)
\begin{align*}
    & P(X\geq 89) \\
    & P\left(\frac{x-68}{14} \geq \frac{89-68}{14}\right) \\
    & P\left(Y\geq \frac{21}{14}\right) = \left(Y \highlight{\geq} \frac{3}{2}\right) = \highlight{1 -} P \left(Y \highlight{\leq} \frac{3}{2}\right) \\
    &= 1 - \Phi(1.5) = 1-0.933=0.067 \\
\end{align*}

Napomena: da bi izracunali $\Phi$ trebamo $\leq$

Rj. Vjerojatnost da je netko dobio odlicaj je 6.7\%.

\end{document}